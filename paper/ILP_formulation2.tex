\documentclass[10pt]{article}
%	options include 12pt or 11pt or 10pt
%	classes include article, report, book, letter, thesis
\usepackage{xcolor}
\usepackage{amsmath}

\title{Generic Physical Designer: ILP Formulation}
\author{Dong Young Yoon}
\date{\today}

\DeclareMathOperator*{\argmin}{arg\,min}
\DeclareMathOperator*{\argmax}{arg\,max}
\newcommand{\dy}[1]{\textcolor{blue}{DY: #1}}

\begin{document}
\maketitle

\section{Setup}
Consider a workload $W$ consisting of $m$ queries (i.e., $W = \{q_1, \dots, q_m\}$) and a set of $n$
physical design structures, $P = \{P_1, \dots, P_n\}$. (e.g., indexes, storage file format for
HDFS-based query engines, etc.)
A configuration is a subset $C_j = \{P_{j1}, P_{j2}, \dots\}$ of physical design structures that
all of the physical design structures in $C_j$ are used by some query $q_i$ in $W$. $C_j$ can be
empty (i.e., $C_j = \{\}$), meaning that there are no extra physical design structures in place.
This follows the definition of \textit{atomic configuration}
\footnote{
`A configuration $C$ is \textit{atomic} for a workload if for some query in the workload
there is a possible execution of the query by the query engine that uses \textit{all} indexes in
$C$'
}
in \cite{chaudhuri1997efficient},
and we are not interested in physical design structures that are not used by any query in $W$.
% \dy{
% For genericity of our physical designer and considering the fact that we might not have the luxury
% of query optimizer letting us know that it uses certain physical design structures or not (depending
% on the target database or the query engine), should/can we also utilize the notion of atomic
% configuration?
% }

Let $Y$ be the set of all $k$ configurations (i.e. $Y = \{C_1, \dots, C_k\}$) that can be constructed
using physical design structures in $P$.
We assume that physical design structures in a configuration $C_j$ are compatible with each other
and can co-exist at the same time
(as some physical design structures could be incompatible to each other.
For instance, a specific storage file format may not support certain types of indexes in a
particular HDFS-based query engine).
The cost of a query $q_i$ using a configuration $C_j$ is $Cost(q_i, C_j)$.

Now, the cost of the workload $Cost(W)$ is defined as:
\begin{equation}
  Cost(W) = \sum_{i=1}^{m} \argmin_{j} Cost(q_i, C_j)
  \label{eq:cost}
\end{equation}

For our ILP formulation, we introduce a binary decision variable $x_{ij}$ (similar to
\cite{dash2011automated, papadomanolakis2007integer}) that is 1 if the query
$q_i$ uses a configuration $C_j$, or 0 otherwise.
With the definition of $x_{ij}$ and under the assumption that a query uses exactly
one configuration, we can rewrite Equation \eqref{eq:cost} as:
\begin{equation}
  Cost(W) = \sum_{i=1}^{m} \sum_{j=1}^{k} x_{ij} \times Cost(q_i, C_j)
\end{equation}
We also define $y_t$ as a binary decision variable that is 1
if the physical design structure $P_t$ is implemented or 0 otherwise.
Note that $x_{ij}$ depends on $y_j$ as a query cannot use a configuration
$C_j$ unless all the physical design structures in $C_j$ are implemented.

In addition, let $E_{st}$ be a binary indicator variable that is 1 if two physical design structures
, $P_s$ and $P_t$, are compatible with each other and can co-exist, or 0 otherwise.
Lastly, we want our set of physical design structures to meet the space requirement such that
each physical design structure $P_t$ takes $s_t$ units of storage space, and
they consume less than $S$ units of storage space in total.

\section{ILP Formulation}
\label{sec:ILP_formulation}

From the setup that we discussed in the previous section, we have the following ILP formulation:
\begin{equation}
  minimize\ Cost(W) = \sum_{i=1}^{m} \sum_{j=1}^{k} x_{ij} \times Cost(q_i, C_j)
  \label{eq:obj}
\end{equation}

Subject to:
\begin{equation}
  0 < \sum_{j=1}^{k} x_{ij} \leq 1,~~~\forall{i}
  \label{eq:const1}
\end{equation}
\begin{equation}
  x_{ij} \leq  y_{t},~~~\forall{i}, \forall{j,t} : P_t \in C_k
  \label{eq:const2}
\end{equation}
\begin{equation}
  y_{k_1} + y_{k_2} - 1 \leq E_{k_1 k_2},~~~\forall{k1, k2}
  \label{eq:const3}
\end{equation}
\begin{equation}
  \sum_{t=1}^{k} s_t \times y_t \leq S
  \label{eq:const4}
\end{equation}

Constraint \eqref{eq:const1} ensures that a query uses exactly one configuration.
Constraint \eqref{eq:const2} guarantees that every physical design structure in a configuration being
used is implemented.
Constraint \eqref{eq:const3} ensures that physical design structures used by different queries are
compatible with each other.
% (\dy{I think the constraint \eqref{eq:const3} needs a refinement as
% I am not 100\% sure that it is correct and valid for ILP})
Constraint \eqref{eq:const4} expresses the storage limit of the target system.

\subsection{With Regression Model}
In our generic physical design, we are going to estimate the benefit of a configuration and the
storage requirement of physical design structure from small subsets of sampled data $D$ from the
original data and regression model $R$.

The only difference that such estimation will impose on the above formulation is with variables
$Cost(q_i, C_j)$ and $s_{t}$.
We denote estimated values of $Cost_(q_i, C_j)$ and $s_{t}$ to be $Cost^\prime(q_i,C_j)$ and
$\sigma_{t}$, respectively.
The cost of a query $q_i$ using a configuration $C_j$ with the sampled data $D$ is
$Cost(q_i, C_j, D)$.

Let $s_t (D)$ be the storage requirement of physical design structure $P_t$ on sampled data $D$.
In addition, let $R(X)$ be the predicted value from the regression model $R$ using $X$ as its input.
We define $Cost^\prime (q_i,C_j)$ and $\sigma_{t}$ obtained from the sampled data $D$ and
the regression model $R$ as follows:
\begin{equation}
  Cost^\prime (q_i,C_j) = R(Cost(q_i, C_j, D))
  \label{eq:regress1}
\end{equation}
\begin{equation}
  \sigma_{t} = R(s_t (D))
\end{equation}

% \dy{Should we just use $Cost(q_i, \{\})$ instead of $R(Cost(q_i, \{\}, D)$ for (\ref{eq:regress1})?}
Now, we can substitute $Cost(q_i,C_j)$ and $s_t$ from the ILP formulation in
Sec \ref{sec:ILP_formulation} with $Cost^\prime (q_i,C_j)$ and $\sigma_{t}$.

\bibliographystyle{abbrv}
\bibliography{ilp}

\end{document}
