\documentclass[10pt]{article}
%	options include 12pt or 11pt or 10pt
%	classes include article, report, book, letter, thesis
\usepackage{xcolor}
\usepackage{amsmath}

\title{Generic Physical Designer: ILP Formulation}
\author{Dong Young Yoon}
\date{\today}

\DeclareMathOperator*{\argmin}{arg\,min}
\DeclareMathOperator*{\argmax}{arg\,max}
\newcommand{\dy}[1]{\textcolor{blue}{DY: #1}}

\begin{document}
\maketitle

\section{Setup}
Consider a workload $W$ consisting of $m$ queries (i.e., $W = \{q_1, \dots, q_m\}$) and a set of $n$
available physical design structures, $P = \{p_1, \dots, p_n\}$
(e.g., indexes, storage file format for HDFS-based query engines, etc. --
we call these as a \textit{structure} and we also say that we \textit{build} such structure
from here on).

We define a particular $j^\text{th}$ subset of $P$ as $S_j$, where $S_j = \{p_{j_1}, p_{j_2}, \dots\}$
and $S_j \subset P$.
$S_j$ can be empty (i.e., $S_j = \{\}$),
meaning that there are no extra structures in place.

Let $Y$ be the set of all $k$ possible subsets of $P$ such that $Y = \{S_1, \dots, S_k\}$,
where each subset $S_j$ is atomic for the workload $W$.
This follows the definition of \textit{atomic configuration}
\footnote{
`A configuration $C$ is \textit{atomic} for a workload if for some query in the workload
there is a possible execution of the query by the query engine that uses \textit{all} indexes in
$C$' -- a \textit{configuration} equals to a set of structures in our formulation.
}
in \cite{chaudhuri1997efficient},
We assume that structures in a subset $S_j$ are compatible with each other
and can co-exist at the same time
(as some structures could be incompatible to each other.
For instance, a specific storage file format may not support certain types of indexes in a
particular HDFS-based query engine).
The cost of a query $q_i$ with every structure in a subset $S_j$ is built
is defined as $Cost(q_i, S_j)$.

Now, the optimal cost of the workload $Cost(W)$ is defined as:
\begin{equation}
  Cost(W) = \sum_{i=1}^{m} \argmin_{j} Cost(q_i, S_j)
  \label{eq:cost}
\end{equation}

For our ILP formulation, we introduce a binary variable $x_{ij}$
that is 1 if every structure in a subset $S_j$ is built and it yields the optimal cost for
a query $q_i$ among all currently available subsets in $Y$
(i.e., subsets with all of their structures are built), or 0 otherwise.

With the definition of $x_{ij}$, we can rewrite Equation \eqref{eq:cost} with respect to the
currently available structures as:
\begin{equation}
  Cost(W) = \sum_{i=1}^{m} \sum_{j=1}^{k} x_{ij} \times Cost(q_i, S_j)
\end{equation}
We define $y_t$ as a binary decision variable that is 1
if the structure $p_t$ is built, or 0 otherwise.
Note that $x_{ij}$ depends on $y_t$ as $x_{ij}$ can be 1 only if every structure in
$S_j = \{p_{j_1}, p_{j_2}, \dots\}$ is built. In other words, $y_t$ must satisfy the following
condition: $ y_t = 1, \forall t \in \{j_1, j_2, \dots\}$

In addition, let $E_{st}$ be a binary indicator variable that is 1 if two structures
, $p_s$ and $p_t$, are compatible with each other and can co-exist, or 0 otherwise.
Lastly, we want our set of physical design structures to meet the space requirement such that
each physical design structure $z_t$ takes $z_t$ units of storage space, and
they consume less than $Z$ units of storage space in total.

\section{ILP Formulation}
\label{sec:ILP_formulation}

From the setup that we discussed in the previous section, we have the following ILP formulation:
\begin{equation}
  minimize\ Cost(W) = \sum_{i=1}^{m} \sum_{j=1}^{k} x_{ij} \times Cost(q_i, S_j)
  \label{eq:obj}
\end{equation}

Subject to:
\begin{equation}
  \sum_{j=1}^{k} x_{ij} = 1,~~~\forall{i}
  \label{eq:const1}
\end{equation}
\begin{equation}
  x_{ij} \leq  y_{t},~~~\forall{i}, \forall{j,t} : p_t \in S_j
  \label{eq:const2}
\end{equation}
\begin{equation}
  y_{k_1} + y_{k_2} - 1 \leq E_{k_1 k_2},~~~\forall{k1, k2}
  \label{eq:const3}
\end{equation}
\begin{equation}
  \sum_{t=1}^{k} z_t \times y_t \leq Z
  \label{eq:const4}
\end{equation}

Constraint \eqref{eq:const1} ensures that the formulation only considers
the optimal cost possible for each query from available structures.
Constraint \eqref{eq:const2} guarantees that every structure in a subset is actually built.
Constraint \eqref{eq:const3} ensures that structures used by different queries are
compatible with each other.
Constraint \eqref{eq:const4} expresses the storage limit of the target system.

\subsection{With Regression Model}
In our generic physical design, we are going to estimate the benefit of structures and the
storage requirement of physical design structure from small subsets of sampled data $D$ from the
original data and regression model $R$.

The only difference that such estimation will impose on the above formulation is with variables
$Cost(q_i, S_j)$ and $z_{t}$.
We denote estimated values of $Cost_(q_i, S_j)$ and $z_{t}$ to be $Cost^\prime(q_i, S_j)$ and
$\sigma_{t}$, respectively.
The cost of a query $q_i$ with a set of structures $S_j$ and the sampled data $D$ is
$Cost(q_i, S_j, D)$.

Let $z_t (D)$ be the storage requirement of physical design structure $p_t$ on sampled data $D$.
In addition, let $R(X)$ be the predicted value from the regression model $R$ using $X$ as its input.
We define $Cost^\prime (q_i,S_j)$ and $\sigma_{t}$ obtained from the sampled data $D$ and
the regression model $R$ as follows:
\begin{equation}
  Cost^\prime (q_i,S_j) = R(Cost(q_i, S_j, D))
  \label{eq:regress1}
\end{equation}
\begin{equation}
  \sigma_{t} = R(z_t (D))
\end{equation}

Now, we can substitute $Cost(q_i,S_j)$ and $z_t$ from the ILP formulation in
Sec \ref{sec:ILP_formulation} with $Cost^\prime (q_i,S_j)$ and $\sigma_{t}$.

\bibliographystyle{abbrv}
\bibliography{ilp}

\end{document}
