\documentclass[10pt]{article}
%	options include 12pt or 11pt or 10pt
%	classes include article, report, book, letter, thesis
\usepackage{xcolor}
\usepackage{amsmath}

\title{Generic Physical Designer: ILP Formulation}
\author{Dong Young Yoon}
\date{\today}

\DeclareMathOperator*{\argmin}{arg\,min}
\DeclareMathOperator*{\argmax}{arg\,max}
\newcommand{\dy}[1]{\textcolor{blue}{DY: #1}}

\begin{document}
\maketitle

\section{Setup}
Consider a workload $W$ consisting of $m$ queries (i.e., $W = \{q_1, \dots, q_m\}$) and a set of $n$
physical design structures, $P = \{P_1, \dots, P_n\}$. (e.g., indexes, storage file format for
HDFS-based query engines, etc.)
A configuration is a subset $C_j = \{P_{j1}, P_{j2}, \dots\}$ of physical design structures that
all of the physical design structures in $C_j$ are used by some query $q_i$ in $W$.
This follows the definition of \textit{atomic configuration}
\footnote{
`A configuration $C$ is \textit{atomic} for a workload if for some query in the workload
there is a possible execution of the query by the query engine that uses \textit{all} indexes in
$C$'
}
in \cite{chaudhuri1997efficient},
and we are not interested in physical design structures that are not used by any query in $W$.
\dy{
For genericity of our physical designer and considering the fact that we might not have the luxury
of query optimizer letting us know that it uses certain physical design structures or not (depending
on the target database or the query engine), should/can we also utilize the notion of atomic
configuration?
}

Let $Y$ be the set of all $k$ configurations (i.e. $Y = \{C_1, \dots, C_k\}$) that can be constructed
using physical design structures in $P$.
We assume that physical design structures in a configuration $C_j$ are compatible to each other and
can co-exist at the same time
(as some physical design structures could be incompatible to each other.
For instance, a specific storage file format may not support certain types of indexes in a
particular HDFS-based query engine).
The cost of a query $q_i$ using a configuration $C_j$ is $Cost(q_i, C_j)$, and $Cost(q_i, \{\})$
denotes the cost of the query with no extra physical design structures in place.

Similar to \cite{papadomanolakis2007integer}, we define the \textit{benefit} of a configuration
$C_j$ for query $q_i$ by:
\begin{equation}
 b_{ij} = \max{(0, Cost(q_i, \{\}) - Cost(q_i, C_j)))}
\end{equation}

Let $y_t$ be a binary decision variable that is 1 if the physical design structure $P_t$ is
implemented or 0 otherwise.
Also, let $x_{ij}$ be a binary decision variable that is 1 if the query $q_i$ uses a configuration
$C_j$, or 0 otherwise. Note that $x_{ij}$ depends on $y_j$ as a query cannot use a configuration
$C_j$ unless all the physical design structures in $C_j$ are implemented.
With the definition of $b_{ij}$ and $x_{ij}$, the benefit for the workload $B(W)$ is:
\begin{equation}
  B(W) = \sum_{i=1}^{m} \sum_{j=1}^{k} b_{ij} \times x_{ij}
\end{equation}
In addition, let $c_{nm}$ be a binary variable that is 1 if two physical design structures
, $P_n$ and $P_m$, are compatible to each other and can co-exist, or 0 otherwise.
Lastly, we want our set of physical design structures to meet the space requirement such that
each physical design structure $P_t$ takes $s_t$ units of storage space, and
they consume less than $S$ units of storage space in total.

\section{ILP Formulation}
\label{sec:ILP_formulation}

\dy{
Note that this formulation is quite similar to the ILP formulation in
\cite{papadomanolakis2007integer}
}
From the setup that we discussed in the previous section, we have the following ILP formulation:
\begin{equation}
  maximize\ B(W) = \sum_{i=1}^{m} \sum_{j=1}^{k} b_{ij} \times x_{ij}
  \label{eq:obj}
\end{equation}

Subject to:
\begin{equation}
  \sum_{j=1}^{k} x_{ij} \leq 1,~~~\forall{i}
  \label{eq:const1}
\end{equation}
\begin{equation}
  x_{ij} \leq  y_{t},~~~\forall{i}, \forall{j,t} : P_t \in C_k
  \label{eq:const2}
\end{equation}
\begin{equation}
  \bigcup C_{k} \in Y,~~~\forall{k} : \sum_{i=1}^{m} x_{ik} \geq 1
  \label{eq:const3}
\end{equation}
\begin{equation}
  \sum_{t=1}^{k} s_t \times y_t \leq S
  \label{eq:const4}
\end{equation}

Constraint (\ref{eq:const1}) ensures that a query uses at most one configuration.
Constraint (\ref{eq:const2}) guarantees that every physical design structure in a configuration being
used is implemented.
Constraint (\ref{eq:const3}) ensures that every configuration used by different queries are compatible
to each other (\dy{I think the constraint (\ref{eq:const3}) needs a refinement as
I am not 100\% sure that it is correct and valid for ILP}).
Constraint (\ref{eq:const4}) expresses the storage limit of the target system.

\subsection{With Regression Model}
In our generic physical design, we are going to estimate the benefit of a configuration and the
storage requirement of physical design structure from small subsets of sampled data $D$ from the
original data and regression model $R$.

The only difference that such estimation will impose on the above formulation is with variables
$b_{ij}$ and $s_{t}$. We denote estimated values of $b_{ij}$ and $s_{t}$ to be $\beta_{ij}$ and
$\sigma_{t}$, respectively.

The cost of a query $q_i$ using a configuration $C_j$ with the sampled data $D$ is
$Cost(q_i, C_j, D)$, and $Cost(q_i, \{\}, D)$
denotes the cost of the query with no extra physical design structures in place.

Let $s_t (D)$ be the storage requirement of physical design structure $P_t$ on sampled data $D$.
In addition, let $R(X)$ be the predicted value from the regression model $R$ using $X$ as its input.
We define $\beta_{ij}$ and $\sigma_{t}$ obtained from the sampled data $D$ and the regression model
$R$ as follows:
\begin{equation}
  \beta_{ij} = R(Cost(q_i, C_j, D)) - R(Cost(q_i, \{\}, D))
  \label{eq:regress1}
\end{equation}
\begin{equation}
  \sigma_{t} = R(s_t (D))
\end{equation}

\dy{Should we just use $Cost(q_i, \{\})$ instead of $R(Cost(q_i, \{\}, D)$ for (\ref{eq:regress1})?}
Now, we can substitute $b_{ij}$ and $s_t$ from the ILP formulation in Sec \ref{sec:ILP_formulation}
with $\beta_{ij}$ and $\sigma_{t}$.

\bibliographystyle{abbrv}
\bibliography{ilp}

\end{document}
